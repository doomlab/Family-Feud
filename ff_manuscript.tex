\documentclass[man]{apa6}
\usepackage{lmodern}
\usepackage{amssymb,amsmath}
\usepackage{ifxetex,ifluatex}
\usepackage{fixltx2e} % provides \textsubscript
\ifnum 0\ifxetex 1\fi\ifluatex 1\fi=0 % if pdftex
  \usepackage[T1]{fontenc}
  \usepackage[utf8]{inputenc}
\else % if luatex or xelatex
  \ifxetex
    \usepackage{mathspec}
  \else
    \usepackage{fontspec}
  \fi
  \defaultfontfeatures{Ligatures=TeX,Scale=MatchLowercase}
\fi
% use upquote if available, for straight quotes in verbatim environments
\IfFileExists{upquote.sty}{\usepackage{upquote}}{}
% use microtype if available
\IfFileExists{microtype.sty}{%
\usepackage{microtype}
\UseMicrotypeSet[protrusion]{basicmath} % disable protrusion for tt fonts
}{}
\usepackage{hyperref}
\hypersetup{unicode=true,
            pdftitle={Gamifying Judgments of Associative Memory},
            pdfauthor={Erin M. Buchanan~\& Abigail Van Nuland},
            pdfkeywords={judgments, word association, metacognition},
            pdfborder={0 0 0},
            breaklinks=true}
\urlstyle{same}  % don't use monospace font for urls
\usepackage{graphicx,grffile}
\makeatletter
\def\maxwidth{\ifdim\Gin@nat@width>\linewidth\linewidth\else\Gin@nat@width\fi}
\def\maxheight{\ifdim\Gin@nat@height>\textheight\textheight\else\Gin@nat@height\fi}
\makeatother
% Scale images if necessary, so that they will not overflow the page
% margins by default, and it is still possible to overwrite the defaults
% using explicit options in \includegraphics[width, height, ...]{}
\setkeys{Gin}{width=\maxwidth,height=\maxheight,keepaspectratio}
\IfFileExists{parskip.sty}{%
\usepackage{parskip}
}{% else
\setlength{\parindent}{0pt}
\setlength{\parskip}{6pt plus 2pt minus 1pt}
}
\setlength{\emergencystretch}{3em}  % prevent overfull lines
\providecommand{\tightlist}{%
  \setlength{\itemsep}{0pt}\setlength{\parskip}{0pt}}
\setcounter{secnumdepth}{0}
% Redefines (sub)paragraphs to behave more like sections
\ifx\paragraph\undefined\else
\let\oldparagraph\paragraph
\renewcommand{\paragraph}[1]{\oldparagraph{#1}\mbox{}}
\fi
\ifx\subparagraph\undefined\else
\let\oldsubparagraph\subparagraph
\renewcommand{\subparagraph}[1]{\oldsubparagraph{#1}\mbox{}}
\fi

%%% Use protect on footnotes to avoid problems with footnotes in titles
\let\rmarkdownfootnote\footnote%
\def\footnote{\protect\rmarkdownfootnote}


  \title{Gamifying Judgments of Associative Memory}
    \author{Erin M. Buchanan\textsuperscript{1}~\& Abigail Van
Nuland\textsuperscript{1}}
    \date{}
  
\shorttitle{JAM GAME}
\affiliation{
\vspace{0.5cm}
\textsuperscript{1} Missouri State University}
\keywords{judgments, word association, metacognition}
\usepackage{csquotes}
\usepackage{upgreek}
\captionsetup{font=singlespacing,justification=justified}

\usepackage{longtable}
\usepackage{lscape}
\usepackage{multirow}
\usepackage{tabularx}
\usepackage[flushleft]{threeparttable}
\usepackage{threeparttablex}

\newenvironment{lltable}{\begin{landscape}\begin{center}\begin{ThreePartTable}}{\end{ThreePartTable}\end{center}\end{landscape}}

\makeatletter
\newcommand\LastLTentrywidth{1em}
\newlength\longtablewidth
\setlength{\longtablewidth}{1in}
\newcommand{\getlongtablewidth}{\begingroup \ifcsname LT@\roman{LT@tables}\endcsname \global\longtablewidth=0pt \renewcommand{\LT@entry}[2]{\global\advance\longtablewidth by ##2\relax\gdef\LastLTentrywidth{##2}}\@nameuse{LT@\roman{LT@tables}} \fi \endgroup}


\DeclareDelayedFloatFlavor{ThreePartTable}{table}
\DeclareDelayedFloatFlavor{lltable}{table}
\DeclareDelayedFloatFlavor*{longtable}{table}
\makeatletter
\renewcommand{\efloat@iwrite}[1]{\immediate\expandafter\protected@write\csname efloat@post#1\endcsname{}}
\makeatother
\usepackage{lineno}

\linenumbers

\authornote{Erin M. Buchanan is an Associate Professor of
Quantitative Psychology at Missouri State University, and Abigail Van
Nuland is a Masters Degree candidate at Missouri State Univeristy.

Enter author note here.

Correspondence concerning this article should be addressed to Erin M.
Buchanan, 901 S. National Ave, Springfield, MO, 65897. E-mail:
\href{mailto:erinbuchanan@missouristate.edu}{\nolinkurl{erinbuchanan@missouristate.edu}}}

\abstract{
One or two sentences providing a \textbf{basic introduction} to the
field, comprehensible to a scientist in any discipline.

Two to three sentences of \textbf{more detailed background},
comprehensible to scientists in related disciplines.

One sentence clearly stating the \textbf{general problem} being
addressed by this particular study.

One sentence summarizing the main result (with the words ``\textbf{here
we show}'' or their equivalent).

Two or three sentences explaining what the \textbf{main result} reveals
in direct comparison to what was thought to be the case previously, or
how the main result adds to previous knowledge.

One or two sentences to put the results into a more \textbf{general
context}.

Two or three sentences to provide a \textbf{broader perspective},
readily comprehensible to a scientist in any discipline.


}

\usepackage{amsthm}
\newtheorem{theorem}{Theorem}[section]
\newtheorem{lemma}{Lemma}[section]
\theoremstyle{definition}
\newtheorem{definition}{Definition}[section]
\newtheorem{corollary}{Corollary}[section]
\newtheorem{proposition}{Proposition}[section]
\theoremstyle{definition}
\newtheorem{example}{Example}[section]
\theoremstyle{definition}
\newtheorem{exercise}{Exercise}[section]
\theoremstyle{remark}
\newtheorem*{remark}{Remark}
\newtheorem*{solution}{Solution}
\begin{document}
\maketitle

We are going to write an introduction here about JAM, association, and
stuff.

\begin{enumerate}
\def\labelenumi{\arabic{enumi})}
\tightlist
\item
  In a traditional free association task, people guess the
  words\ldots{}would the numbers help? (a v b)
\item
  In a JAM task with multiple guesses, they would normally see the
  words, would expect them to be able to guess better if they know it's
  descending order, so the D condition would match the normal JAM slope
  and C should be better (this answers the bounded question).
\item
  For the non-numbers condition, see if singles or groups are better.
  Shouldn't more heads be better than one?
\item
  For the normal JAM task, see if groups are better than singles at
  guessing the numbers.
\end{enumerate}

Judgments of associative memory are notoriously poor (Buchanan, 2009;
Maki, 2007), where participants over estimate the relationship between
word pairs. In the judgment task, participants are given two words
(LOST-FOUND) and asked to rate how many people out of a 100 would list
the second word if given the first word. Participants cannot tell the
difference between low and high frequency pairs and tend to judge pairs
higher than they should. The psycholinguistics lab has tried to correct
these memory judgments by giving participants various instructions
(Buchanan \& Maki, in preparation), changing the scales for judgments
(Buchanan, data analysis), changing the judgment type (Maki \& Buchanan,
in preparation), and having participants judge their own ratings over
time (Buchanan, data collection). These manipulations have shown a small
effect on judgment ability, mainly to reduce the overall bias to select
very large numbers. The current protocol will examine if judgments are
more accurate when the experimental task is more interesting and
engaging. We will be using the game show Family Feud as the experimental
procedure because the game show closely matches the judgments of memory
paradigm currently used.

\section{Method}\label{method}

\subsection{Participants}\label{participants}

Participants were recruited from the Psychology Department at a large
Midwestern university using an online participant management system.
They were given course credit for their participation in the study.
Participants were assigned to work alone (247) or in groups (184 groups,
418) with a total of 431 sessions and 665 completing the study. No other
demographic information was collected.

\subsection{Materials}\label{materials}

Stimuli were selected from the University of South Florida Free
Association Database ({\textbf{???}}). Free association norms were
created by asking participants to list the first word that comes to mind
given a cue word. For example, when shown the cue \emph{lost}, many
participants will then list the target word \emph{found}. These
responses were collected over many participants and use the create a
probability of eliciting the target word given a cue word called forward
strength (FSG). The purpose of this study was to examine free
association and judgments of associative memory over multiple cues, and
therefore, cues with at least four matching target words were selected.
For example, when given the cue \emph{conditioner}, participants might
list \emph{shampoo} (FSG = .455), \emph{hair} (FSG = .325), and
\emph{air} (FSG = .110). Cues with at least one target in each of the
forward strength ranges of .40-.60, .20-.40, and \textless{}.20 were
selected. Eight cue words were selected with an average of 5.00 target
words paired with each cue (\emph{SD} = 0.93). The average forward
strength for high strength targets was 0.44 (\emph{SD} = 0.03), medium
strength targets 0.26 (\emph{SD} = 0.04), and low strength targets 0.06
(\emph{SD} = 0.05). The complete materials can be found online at OSF
LINK.

\subsection{Procedure}\label{procedure}

The eight cue-target lists were assembled into a powerpoint document
that emulated a game of Family Feud. Family Feud is a game show that
gives contestants category labels and asks them to list the \ldots{}

An experimenter will be with the participant at all times to keep score
and be the game show host. The rules of the game will be explained to
the participant as follows: \enquote{You will be playing Family Feud for
your experimental credit today. We asked 100 people to say the first
thing they thought of when given each category you are going to see
today. For example, when we gave people the word}steak" many of them
listed \enquote{sauce, cow, sirloin}. In the following rounds, you will
guess what words people listed. (OR you will guess the number of people
who listed each word) You will receive points for your correct guesses.
Try to beat the high lab score!"

four different versions two of the versions included single participants
versus groups of participants (a and d)

\subsubsection{Version A}\label{version-a}

guess the words, no numbers present three guesses wrong and we moved on
most matches a traditional free association task

\subsubsection{Version B}\label{version-b}

guess the words, numbers were present three guesses wrong and we moved
on

\subsubsection{Version C}\label{version-c}

guess the numbers, words were in numerical order 1 guess to get it right
and they were considered right if it was within five points

\subsubsection{Version D}\label{version-d}

guess the numbers, words were NOT in numerical order 1 guess to get it
right and they were considered right if it was within five points most
matches a JAM task

Participants will be in one of three conditions: • Regular Family Feud:
This condition is played like the game show. Participants are given the
category label and asked to guess four words that people listed. They
are given three strikes at guessing before moving onto the next round.
When they guess a correct word, they are shown the word and points on a
computer screen. The experimenter will keep score. • Numbered Family
Feud: This condition is the same as above, with one exception.
Participants will be able to see the number of people who listed each
word next to the ? on each category label. This condition will examine
if participants have an easier or harder time guessing with the scores
listed. • Reverse Family Feud: This condition is played where
participants are required the guess the number of people who listed each
word under a category, mirroring the judgments of associative memory
task previously used by the researcher. They will be told to guess
within 10-20 people of the words (this number will be pilot tested by
the lab assistants to find the range that allows participants to
\enquote{win}).\\
The research assistants will pilot test all three conditions to come up
with high scores for the game. The scores will be biased lower, so that
participants can almost always score in the high score range. These
scores will be posted in the lab for participants to try to beat. The
first experiment will test participants individually, to eliminate any
social facilitation and create a control group. The second experiment
will test participants in pairs/threes to analyze judgments in groups.

\subsection{Data analysis}\label{data-analysis}

Going to talk here about MLM and Log regression as the analyses of
choice because they are the most appropriate.

\section{Results}\label{results}

\subsection{Hypothesis 1}\label{hypothesis-1}

• Version A versus B -- would answer if the number helped them guess or
not o Multilevel log regression o Random: participant, word? o IVs:
Version, Points (sort of L, M, H words) o DV: correct (1) versus
incorrect (0)

\subsection{Hypothesis 2}\label{hypothesis-2}

• Version C versus D -- would answer if we are better if they are in
order or not o Multilevel regression o Random: participant, word? o IVs:
Version, Points (sort of L, M, H words) o DV: Score they guessed

\subsection{Hypothesis 3}\label{hypothesis-3}

• Version A singles versus Groups o Multilevel log regression o Random:
participant, word? o IVs: Singles/Groups, Points (sort of L, M, H words)
o DV: correct (1) versus incorrect (0)

\subsection{Hypothesis 4}\label{hypothesis-4}

• Version D singles versus Groups o Multilevel regression o Random:
participant, word? o IVs: Singles/groups, Points (sort of L, M, H words)
o DV: Score they guessed

\section{Discussion}\label{discussion}

\newpage

\section{References}\label{references}

\begingroup
\setlength{\parindent}{-0.5in} \setlength{\leftskip}{0.5in}

\hypertarget{refs}{}

\endgroup


\end{document}
